Infectious diseases are notorious for their complex dynamics, which make it difficult to fit models to test hypotheses.
Methods based on state-space reconstruction have been proposed to infer causal interactions in noisy, nonlinear dynamical systems.
These ``model-free'' methods are collectively known as convergent cross-mapping (CCM).
Although CCM has theoretical support, natural systems routinely violate its assumptions.
To identify the practical limits of causal inference under CCM, we simulated the dynamics of two pathogen strains with varying interaction strengths.
The original method of CCM is extremely sensitive to periodic fluctuations, inferring interactions between independent strains that oscillate with similar frequencies.
This sensitivity vanishes with alternative criteria for inferring causality.
However, CCM remains sensitive to high levels of process noise and changes to the deterministic attractor.
This sensitivity is problematic because it remains challenging to gauge noise and dynamical changes in natural systems, including the quality of reconstructed attractors that underlie cross-mapping.
We illustrate these challenges by analyzing time series of reportable childhood infections in New York City and Chicago during the pre-vaccine era.
We comment on the statistical and conceptual challenges that currently limit the use of state-space reconstruction in causal inference.
